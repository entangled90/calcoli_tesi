\documentclass[a4paper,12pt]{report}
\usepackage[italian]{babel}
\usepackage{amsmath}
%%\usepackage{mathtools}
\usepackage{amstext}
\usepackage{amssymb}
\usepackage{amsthm}
\usepackage{fullpage}
%%\usepackage{mathrsfs}
\usepackage[utf8]{inputenc}
%%\usepackage{natbib}
\usepackage{jheppub}
\usepackage{verbatim}
\title{Note per riduzione dualità Kutasov-Schwimmer 4d $\rightarrow$ 3d}
\author{Carlo Sana}
\makeatletter
\renewcommand*\env@matrix[1][*\c@MaxMatrixCols c]{%
  \hskip -\arraycolsep
  \let\@ifnextchar\new@ifnextchar
  \array{#1}}
\makeatother
\newcommand{\sign}{\mbox{sign}}
 \begin{document}
\maketitle
\newpage
\section{Teoria Elettrica}
 L'indice superconforme della teoria elettrica di Kutasov-Schwimmer ( $ SU(N_C)  \times SU(N_f)_L)\times SU(N_f)_R)\times  U(1)_B  $) è dato da (vedi \citep{Dolan:2008qi})
 \begin{equation}
\begin{aligned}
& i_E (p,q,v, y,\tilde y ,z) = \\ & -\bigg({p \over 1-p}+{q \over 1-q} -{1 \over (1-p)(1-q)}
\big((p\,q)^{s}- (p\,q)^{1-s}\big)
\bigg) \big( p_{N_c}( z)\, p_{N_c}(z^{-1})-1\big ) \\ 
&{} +{1\over (1-p)(1-q)}\bigg((p\,q)^{{1\over 2}r} \, v \, p_{N_f}(y)\, p_{N_c}(z)
- (p\,q)^{1- {1\over 2}r} \, {1 \over v}\, p_{{N_f}}(y^{-1})\, p_{{N_c}}(z^{-1})\\
&\qquad\qquad\qquad\qquad
+ (p\,q)^{{1\over 2}r}\, {1\over v}\, p_{{N_f}}({\tilde y}\,) \, p_{{N_c}}(z^{-1})
- (p\,q)^{1- {1\over 2}r} \, v\, p_{N_f}({\tilde y}^{-1})\, p_{N_c}(z)\bigg) \,\\
\end{aligned}
\end{equation}
I polinomi sono definiti come
$$
 p_{N_c}({x}) = \sum_i^{N_c} x_i \qquad p_{N_c}({x^{-1}}) =  \sum_i^{N_c} { 1 \over x_i  }
$$
Esplicitando i polinomi si ottiene 
\begin{align*}
& i_E (p,q,v, y,\tilde y ,z) = \\ & -\bigg({p \over 1-p}+{q \over 1-q} -{1 \over (1-p)(1-q)}
\big((p\,q)^{s}- (p\,q)^{1-s}\big)
\bigg) \big(\sum_{1 \leq i,j \leq N_c} {z_i \over z_j} -1\big ) \\ 
&{} +{1\over (1-p)(1-q)}\sum_{i = 1}^{N_f} \sum_{j = 1}^{N_c}\bigg( (p\,q)^{{1\over 2}r} \, v \, y_i \, z_j
- (p\,q)^{1- {1\over 2}r} \, {1 \over v}\,y_i^{-1} \, z_j^{-1})\\
&\qquad\qquad\qquad\qquad
+ (p\,q)^{{1\over 2}r}\, {1\over v}\, \tilde y_i \, z_j^{-1}
- (p\,q)^{1- {1\over 2}r} \, v\, \tilde y_i^{-1}\, z_j\bigg) \,\\
\end{align*}
riscalando $ (pq)^{ \frac{1}{2} r }v y \rightarrow y$ e  
$ (pq)^{- \frac{1}{2} r } v \tilde y \rightarrow \tilde y$  :
\begin{equation}
\begin{aligned}
&i_E(p,q,y,{\tilde y},y) =\\ 
&{} -\bigg({p\over 1-p}+{q\over 1-q } -{1 \over (1-p)(1-q)}
\big((p\,q)^{s}- (p\,q)^{1-s}\bigg)
\bigg(\sum_{1\leq i,j\leq N_c}z_i/z_j-1\bigg)\cr\\ 
&{}+{1\over
(1-p)(1-q)}\sum_{i=1}^{N_f}\sum_{j=1}^{N_c}\Big(\big (y_i-p\,q\,{\tilde y}_i \big )z_j
+ \big ({\tilde y}_i{\!}^{-1}-p\,q\,y_i{\!}^{-1} \big )z_j{\!}^{-1}\Big)\, 
\end{aligned}
\end{equation}
dove $R_q$ e $R_X$ sono le R-cariche della materia (nella fondamentale e nell'aggiunta).
\begin{equation}
R_Q =1-{2\over k+1}{N_c\over N_f} \, , \qquad s = {1 \over k+1} = {1 \over 2 } R_X  \\ 
\end{equation}
Si nota che questa scelta di R-Carica è stata fatta imponendo che la R-simmetria sia non anomala in 4D. In 3D la R-simmetria si può mixare con le altre simmetrie e le cariche non sono più vincolate in questo modo (ok?).\\
 L'indice superconforme è definito come:
 \begin{equation*}
 I_E (p,q,v, y,\tilde y) = \int_{SU(N_c)}\, d \mu(z) \, \exp \bigg( \sum_{n=1}^{\infty} {1 \over n} i_E (p^n,q^n,v^n, y^n,\tilde y^n,z^n) \bigg)
 \end{equation*}
 L'integral sul gruppo $SU(N_c)$ può essere scritto come integrale sulla Cartan del gruppo attraverso:
 \begin{equation}
  \int_{SU(N_c)}\, d \mu(z) \, f(z) = {1 \over N_c!} \int_{T^{N_c-1}} \prod_{i=1}^{N_c } 
  { d z_i \over { 2 \pi i z_i} } \Delta (z) \Delta (z^{-1}) f(z) \bigg \rvert_{\prod_{i=1}^{N_c} z_i = 1} 
 \end{equation}
 e dove $\Delta (z) $ è il determinante di Vandermonde:
 $$
 \Delta (z ) \, = \, \prod_{ \overset{1 \leq i,j< \leq N_c} { i \neq j}}^{N_c} ( z_i - z_j) \, = \,  \prod_{ \overset{1 \leq i,j< \leq N_c} { i \neq j}}^{N_c}\bigg ( 1 - {z_i \over  z_j } \bigg) \, z_j = \,  \prod_{ \overset{1 \leq i,j< \leq N_c} { i \neq j}}^{N_c}\bigg ( 1 - {z_i \over  z_j } \bigg)
 $$
 L'ultima equivalenza è dovuta al vincolo $ \prod_{i=1}^{N_c} z_i = 1$.
 \\
 Ogni termine dell'indice superconforme a singola particella $i_E$ si fattorizza nell'indice ``completo'' $I_E$ essendo all'interno di un esponenziale.\\
s \subsubsection{Contributo dalla parte vettoriale}
 Abbiamo per la parte vettoriale (nell'aggiunta):
 \begin{align*} &\exp \bigg( \sum_{n=1}^{\infty} {1 \over n} i_E^{Vett} (p^n,q^n,z^n) \bigg) \overset{def}{=} \\
 &\exp \bigg( \sum_{n=1}^{\infty} - {1 \over n} \, \bigg( {p^n \over 1-p^n} + {q^n \over 1-q^n} \bigg)  \bigg(\bigg( \sum_{ 1 \leq i,j \leq N_c}  {z_i^n \over z_j^n}  \bigg)- 1 \bigg) \bigg) \, = \\
 =&\exp \bigg( \sum_{n=1}^{\infty} - {1 \over n} \, \bigg( {p^n \over 1-p^n} + {q^n \over 1-q^n} \bigg)  \bigg( \bigg( \sum_{\overset{ 1 \leq i,j \leq N_c}{i \neq j}}  {z_i^n \over z_j^n} \bigg) + \bigg( \sum_{i=1}^{N_c} 1 \bigg)- 1 \bigg) \bigg) \, = \\
  =&\exp \bigg( \sum_{n=1}^{\infty} - {1 \over n} \, \bigg( {p^n \over 1-p^n} + {q^n \over 1-q^n} \bigg)  \bigg( \bigg( \sum_{\overset{ 1 \leq i,j \leq N_c}{i \neq j}}  {z_i^n \over z_j^n} \bigg) + \bigg( N_c \bigg)- 1 \bigg) \bigg) \, = \\
    =&\bigg[ \exp \bigg( \sum_{n=1}^{\infty} - {1 \over n} \, \bigg( {p^n \over 1-p^n} + {q^n \over 1-q^n} \bigg)  \bigg( \bigg( \sum_{\overset{ 1 \leq i,j \leq N_c}{i \neq j}}  {z_i^n \over z_j^n} \bigg) \bigg)\bigg] \\
	&    \exp \bigg( \sum_{n=1}^{\infty} - {1 \over n} \, \bigg( {p^n \over 1-p^n} + {q^n \over 1-q^n} \bigg)  \bigg( N_c - 1 \bigg) \bigg) \, = \\
=& \bigg[\prod_{\overset{ 1 \leq i,j \leq N_c}{i\neq j}} \exp \bigg( \sum_{n=1}^{\infty} - {1 \over n} \, \bigg( {p^n \over 1-p^n} + {q^n \over 1-q^n} \bigg)   {z_i^n \over z_j^n}\bigg) \bigg] \,  
\bigg[ \exp \bigg( \sum_{n=1}^{\infty} - {1 \over n} \, \big( {p^n \over 1-p^n} + {q^n \over 1-q^n} \big) \bigg) \bigg] ^{N_c-1}
\end{align*}

\begin{equation}
\bigg[\prod_{\overset{ 1 \leq i,j \leq N_c}{i\neq j}} \exp \bigg( \sum_{n=1}^{\infty} - {1 \over n} \, i_E^V(p^n,q^n)  {z_i^n \over z_j^n}\bigg) \bigg] \,  
\bigg[ \exp \bigg( \sum_{n=1}^{\infty} - {1 \over n} \,  i_E^V(p^n,q^n) \bigg) \bigg] ^{N_c-1}
\end{equation}

Da questi termini si ottengono le funzioni $\Gamma_e$ attraverso le seguenti identità non banali:

\begin{align}
\exp \bigg( \sum_{n=1}^{\infty} {1 \over n} i_E^{V} (p^n,q^n) \bigg) \bigg( z^n + z^{-n} \bigg) \,& = 
\,  \frac{ \theta ( z;p) \theta (z;q) }{ 1-z^2 } \\
& =  \, \frac{1}{ ( 1 - z ) ( 1 - z^{-1}) \Gamma_e( z;p,q) \Gamma_e( z^{-1}; p,q)} \, 
\label{Gamma_eVectorAdjoint} \\ 
\end{align}

\begin{align}
  \exp \bigg( \sum_{n=1}^{\infty} {1 \over n} i_E^{V} (p^n,q^n) \bigg)  &= (p;p) (q;q) \,
  \label{Gamma_eVector} \\
 \mbox{dove} \quad i_E^V(p^n,q^n) \, &= \, - \bigg( {p^n \over 1-p^n} + {q^n \over 1-q^n} \bigg) 
\end{align}

 Le funzioni ipergeometriche sono definite attraverso:
 $$
 \begin{aligned}
   \Gamma_e (z;p,q) &= \prod_{i,j \geq 0} \frac{ 1 - y^{-1} p^{j+1} q^{k+1}}{ 1 - y p^j q^k}\\
  \quad \theta(z;p) &= \prod_{j \geq 0 } (1- z p^j) ( 1- z^{-1}p^{j+1}) \\
  (x;p) &= \prod_{j \geq 0} ( 1- xp^j)
  \end{aligned}
$$

L'identità \ref{Gamma_eVector} si utilizza per l'ultimo termine dell'indice e applicandola direttamente si trova:
$$
	(p;p)^{N_c-1} ( q;q)^{N_c-1}
$$
 Prima di utilizzare l'identità \ref{Gamma_eVectorAdjoint} è necessario considerare che:
 $$
 \prod_{ \overset{ 1 \leq i,j \leq N_c}{ i\neq j }} {z_i^n \over z_j^n} \,  = \, 
  \prod_{ 1 \leq i < j \leq N_c}\bigg( {z_i^n \over z_j^n} + {z_j^n \over z_i^n}   \bigg)
 $$
 A questo punto identificando $ {z_i \over z_j} = z$ si applica l'identità \ref{Gamma_eVectorAdjoint} per ogni termine della produttoria e si ottiene:
\begin{align*}
\prod_{ 1 \leq i < j \leq N_c} \exp \bigg( \sum_{n=1}^{\infty} - {1 \over n} \, \bigg( {p^n \over 1-p^n} + {q^n \over 1-q^n} \bigg)  \bigg( {z_i^n \over z_j^n} + {z_j^n \over z_i^n}\bigg) \bigg) \, =\\
\prod_{ 1 \leq i < j \leq N_c} \exp \bigg( \sum_{n=1}^{\infty} - {1 \over n} \, i_E^V(p^n,q^n)  \bigg( {z_i^n \over z_j^n} + {z_j^n \over z_i^n}\bigg) \bigg) 
\\
 \prod_{ 1 \leq i < j \leq N_c}   \, \frac{ 1 } { \big( 1 -{ z_i \over z_j} \big) \big ( 1 - { z_j \over z_i} \big) \Gamma_e( {z_i \over z_j};p,q) \Gamma_e({z_j \over z_i}; p,q)  }
\end{align*}
 Mettendo insieme i contributi per la parte vettoriali otteniamo:
 \begin{align*}
& (p;p)^{N_c-1} ( q;q)^{N_c-1} \, \prod_{ 1 \leq i < j \leq N_c} \frac{ 1 }{ \big( 1 -{ z_i \over z_j} \big) \big ( 1 - { z_j \over z_i} \big) \Gamma_e( {z_i \over z_j};p,q) \Gamma_e({z_j \over z_i}; p,q)  } \\
& (p;p)^{N_c-1} ( q;q)^{N_c-1} \, \frac{1}{ \Delta(z) \Delta (z^{-1})} \, \prod_{ 1 \leq i < j \leq N_c} \frac{ 1 }{ \Gamma_e( {z_i \over z_j};p,q) \Gamma_e({z_j \over z_i}; p,q)} 
\end{align*}
\subsubsection{Contributo della materia nell'aggiunta}
Per il calcolo del contributo dato dalla materia nella rappresentazione aggiunta è necessario utilizzare l'identità matematica:
$$
\Gamma_e(z;p,q) \, = \, \exp \bigg( \sum_{n=1}^{\infty} {1 \over n }
 \frac{ z^n - \big( {pq \over z} \big)^n }{(1-p^n) ( 1-q^n)} \bigg)
 \label{Gamma_eChiral}
$$
L'indice a singola particella dato da questo campo è dato da:
$$
\i_E^{Adj}(p,q,z) = {1 \over (1-p)(1-q)}
\big((p\,q)^{s}- (p\,q)^{1-s}\big) \bigg( \bigg(  \sum_{ 1 \leq i,j \leq N_c}  {z_i\over z_j}  \bigg)- 1 \bigg)
$$
L'espressione da calcolare è
\begin{equation}
I_E^{Adj}(p,q,z)  = \exp \bigg(  \sum_{n=1}^{\infty} {1 \over n } \i_E^{Adj}(p^n,q^n,z^n) \bigg)
\end{equation}

Come è stato fatto per la parte vettoriale, si spezza la serie, sommando solo sulle coppie:
\begin{align*}
 &\bigg( \sum_{ 1 \leq i,j \leq N_c}  {z_i^n \over z_j^n}  \bigg)- 1  = \\
 &  \bigg( \sum_{ 1 \leq i< j \leq N_c}  {z_i^n \over z_j^n} + {z_j^n \over z_i^n}  \bigg) + N_c -1 
\end{align*}
Si arriva quindi a
\begin{align*}
I_E^{Adj}(p,q,z)  = \exp \bigg[  \sum_{n=1}^{\infty} {1 \over n } {1 \over (1-p^n)(1-q^n)}
\big((p \, q)^{sn}- (p\,q)^{(1-s)n}\big) \bigg( \bigg(  \sum_{ \overset{1 \leq i,j \leq N_c}{i \neq j}}  {z_i^n\over z_j^n} + { z_j^n \over z_i^n}  \bigg)+N_c- 1 \bigg)\bigg]
\end{align*}
Come fatto precedentemente, si calcolano separatamente i termini che dipendono da z da quelli che non ne dipendono.\\
Per calcolare l'indice superconforme è necessario calcolare il \emph{plethystic exponential} come negli altri casi. Per i termini non dipendenti da ${z_i \over z_j}$ è dato da:
$$ 
\begin{aligned}
&\exp \bigg( (N_c -1) \sum_{n=1}^{\infty} {1 \over n}  \frac{ (pq)^{sn} - (pq)^{(1-s)n}}{(1-p^n)(1-q^n)} \bigg) =
\exp \bigg( (N_c-1) \sum_{n=1}^{\infty} {1 \over n}  \frac{ (y)^{n} - ({pq \over y})^{n}}{(1-p^n)(1-q^n)} \bigg)\\
&\mbox{Avendo posto} \quad (pq)^s = y
\end{aligned}
$$
L'identità \ref{Gamma_eChiral} si applica immediatamente ai termini indipendenti da $z_i$ e si ottiene un contributo pari a:
$$
	\Gamma_e( (pq)^s;p,q)^{N_c-1}
$$
Per i termini dipendenti da $z_i$ consideriamo il numeratore dell'esponente (il denominatore non viene alterato)
\begin{align*}
& ((pq)^{sn} - (pq)^{(1-s)n}) \bigg ( {z_i^n\over z_j^n} + { z_j^n \over z_i^n} \bigg) \\
\intertext{Riarrangiando i 4 termini}
& \bigg( (pq)^{sn} {z_i^n\over z_j^n} - (pq)^{(1-s)n}{ z_j^n \over z_i^n} \bigg)  + \bigg( (pq)^{sn} { z_j^n \over z_i^n} - (pq)^{(1-s)n}{z_i^n\over z_j^n} \bigg)
\end{align*}
il cambio di variabile da effettuare è
$$
	 y = (pq)^s {z_i \over z_j} \quad y'= (pq)^s {z_j \over z_i} 
$$
per i termini nella prima e seconda parentesi rispettivamente.
A questo punto si applica l'identità \ref{Gamma_eChiral} utilizzando le variabili $y,y'$ e il contributo è pari a
$$
\prod_{1\leq i<j\leq N_c} \Gamma_e \bigg( (pq)^s {z_i \over z_j} \bigg) \Gamma_e \bigg( (pq)^s {z_j \over z_i}\bigg) 	
$$
Riassumento il contributo dato dalla materia nell'aggiunta è:
$$
	\Gamma_e( (pq)^s;p,q)^{N_c-1} \prod_{1\leq i<j\leq N_c} \Gamma_e \bigg( (pq)^s {z_i \over z_j} \bigg) \Gamma_e \bigg( (pq)^s {z_j \over z_i}\bigg) 
$$
\subsubsection{Contributo materia nella fondamentale}
Per questo campo è necessario calcolare (dopo il riscalamento di $y$ e $\tilde y$):
\begin{align*}
\prod_{\overset{ 1 \leq j \leq N_c}{ 1 \leq i \leq N_f}} \exp \bigg[ \sum_{n=1}^{\infty} - {1 \over n} \, {1 \over (1-p^n)( 1-q^n)}
\bigg[ \bigg( (y_i z_j)^n - \big({ pq \over y_i z_j}\big)^n \bigg) + \bigg( {1 \over  (\tilde{y_i} z_j)^n } - \big( { pq \tilde y_i z_j} \big)^n\bigg) \bigg] \bigg]
\end{align*}
Identificando $ t = y_i z_j$ e $ t' = (\tilde y_i z_j)^{-1}$ con gli argomenti delle $\Gamma_e$ nell'identità \ref{Gamma_eChiral} si può scrivere il contributo della materia nella fondamentale applicando direttamente l'identità (separatamente per i due termini nelle parentesi) (ricordando il rescaling iniziale):
$$
\prod_{\overset{ 1 \leq j \leq N_c}{ 1 \leq i \leq N_f}} \Gamma_e ( (pq)^{R_Q \over 2} v y_i z_j )
\Gamma_e ( (pq)^{R_Q \over 2} v^{-1} {\tilde y_i}^{-1} z_j^{-1})
$$
\subsection{Formula per Indice superconforme Kutasov-Schwimmer 4d}
Mettendo insieme tutti i contributi e aggiungendo anche l'integrazione sul gruppo di gauge si ottiene l'espressione finale per l'indice superconforme
\begin{align*}
&I_{El} ( p,q,y,\tilde y , v) = \\
 & { 1 \over N_c !} (p;p)^{N_c-1} ( q;q)^{N_c-1} \, \Gamma_e( (pq)^s;p,q)^{N_c-1} \\
&\int_{T^{N_c-1}} \bigg( \prod_{i=1}^{N_c } { d z_i \over { 2 \pi i z_i} } \bigg) \delta \bigg( \prod_{i=1}^{N_c} z_i - 1 \bigg) 
\prod_{ 1 \leq i < j \leq N_c} \frac{ \Gamma_e \big( (pq)^s {z_i \over z_j} \big) \Gamma_e \big( (pq)^s {z_j \over z_i} \big) }{ \Gamma_e( {z_i \over z_j};p,q) \Gamma_e({z_j \over z_i}; p,q)} \\
& \prod_{ 1 \leq j \leq N_c} \prod_{ 1 \leq i \leq N_f} \Gamma_e ( (pq)^{R_Q \over 2} v y_i z_j )
\Gamma_e ( (pq)^{R_Q \over 2} v^{-1} {\tilde y_i}^{-1} z_j^{-1})
\end{align*}
Il determinante di Vandermonde dovuto alla riduzione dell'integrazione alla Cartan si è cancellato con il contributo dato dalla parte vettoriale.
\subsection{Riduzione dell'indice alla funzione di partizione}

Parametrizzando i vari "potenziali chimici" si può calcolare la funzione di partizione, nel limite $ r \rightarrow 0$.\\
$$
\begin{aligned}
p = e^{ 2 \pi i r  \omega_1 } \, \, \,  q &= e^{ 2 \pi i r \omega_2 } \,  \, \, z_i = e^{ 2 \pi i r \sigma_i } \\ 
\, \, y_a = e^{ 2 \pi i r m_a } \, \,  \, 
y_a &= e^{ 2 \pi i r  {\tilde m_a} } \, \, \,  v = e^{ 2 \pi i r m_B}
\label{fugacities_redefined}
\end{aligned}
$$

Identità fondamentale per calcolare questo limite è la seguente (cfr \citep{vanDeBult:2007} pag 30)
\begin{align*}
&\lim_{r \rightarrow 0^+} \Gamma_e (e^{ 2 i r z}; e^{ i  r \omega_1} , e^{i r  \omega_2})
 e^{\frac{ i \pi^2 }{12  r \omega_1 \omega_2 } ( 2 z - \omega_1 -\omega_2)} =\\
&\lim_{r \rightarrow 0^+} \Gamma_e (e^{ 2 i r z}; e^{ i  r \omega_1} , e^{i r  \omega_2}) 
 e^{\frac{ i \pi^2 }{6 r \omega_1 \omega_2 } (  z - \omega )} = \Gamma_h(z;\omega_1 , \omega_2)
\end{align*}
con $ \omega = {1 \over 2 } ( \omega_1 + \omega_2)$.
Si possono riscalare le variabili in modo da sistemare il fattore di $\pi$ all'esponente:
$$
 z \rightarrow \pi z \quad  \omega_1 \rightarrow  \pi \omega_1 \quad  \omega_2 \rightarrow  \pi \omega_2 
$$
Si ottiene:
\begin{align*}
&\lim_{r \rightarrow 0^+} \Gamma_e (e^{ 2 i \pi r z}; e^{ i \pi  r \omega_1} , e^{i \pi r  \omega_2}) =
 e^{\frac{ - i \pi^2 }{6 r  (\pi \omega_1) (\pi \omega_2) } ( (\pi z) - (\pi \omega))}\Gamma_h(\pi z; \pi \omega_1 ,\pi \omega_2) =   e^{\frac{ - i \pi }{6 r \omega_1 \omega_2 } (  z - \omega )} \, \Gamma_h ( z; \omega_1 , \omega_2 ) \\
\end{align*}
Considerando la proprietà di rescaling di $\Gamma_h$: la sua definizione infatti è ( cft \citep{vanDeBult:2007} 2.2.4):
\begin{align*}
 \Gamma_h ( z;\omega_1, \omega_2) =& \exp \bigg( \pi i \frac{(2z-\omega_1 - \omega_2)^2}{8 \omega_1 \omega_2 } - \pi i \frac{(\omega_1^2 + \omega_2^2)}{ 24 \omega_1 \omega_2} \bigg) \\
 & \frac{ (\exp( -2 \pi i (z-\omega_2)/ \omega_1 ); \exp( 2 \pi i \omega_2 / \omega_1 ))_{\infty}}
 { (\exp( -2 \pi z / \omega_2 ); \exp( - 2 \pi i \omega_1 / \omega_2))_{\infty}}\\
 & = \Gamma_h ( \pi z; \pi \omega_1, \pi \omega_2)
\end{align*}

\subsubsection{Funzione di partizione}
Intanto scrivo le parti non divergenti date dal limite per $ r \rightarrow 0$ utilizzando le ridefinizioni delle fugacità in \ref{fugacities_redefined}. Di seguito il calcolo dei vari limiti (ricordare che $ s = \frac{\Delta_X}{2}$
\begin{align*}
&\lim_{r \rightarrow 0^+} \Gamma_e((pq)^{\frac{\Delta_X}{2}})^{N_c -1} =
\lim_{r \rightarrow 0^+ } \Gamma_e( e^{2 \pi i r  {\Delta_X \over 2} (\omega_1 + \omega_2)})^{N_c-1} = \lim_{r \rightarrow 0^+ } \Gamma_e( e^{2 \pi i r  \Delta_X \omega })^{N_c-1} = \\
&=\big[ e^{- \frac{i \pi} {6 r \omega_1 \omega_2 }  ( \omega \Delta_x - \omega) }\big]^{N_c-1} \Gamma_h ( \omega \Delta_X ; \omega_1, \omega_2)^{N_c-1}
%%%%%%%%%%%%%%%%%%%%%%%%
\end{align*}
 \begin{align*}
\lim_{r \rightarrow 0^+} & \Gamma_e\big( (pq)^{\frac{\Delta_X}{2}} \big( { z_i \over z_j} \big) \big) \Gamma_e \big( (pq)^{\frac{\Delta_X}{2}} \big( { z_j \over z_i} \big) \big) = \\
&= \Gamma_e\big( e^{2 \pi i r \frac{\Delta_X}{2} (\omega_1 + \omega_2)} e^{ 2 \pi  i r (\sigma_i - \sigma j)} \big) \big) \Gamma_e \big( e^{ 2 \pi i r \frac{\Delta_X}{2} (\omega_1 + \omega_2)}  e^{ 2 \pi i r (\sigma_j - \sigma_i)} \big) =  \\
&= \Gamma_e\big( e^{2 \pi  i r (\Delta_X \omega +\sigma_i - \sigma j} \big) \Gamma_e \big(  e^{2 \pi  i r (\Delta_X \omega +\sigma_j - \sigma_i} \big) \\
 & = e^{- \frac{i \pi} {6 r \omega_1 \omega_2 }  ( 2 \omega \Delta_x + (\sigma_i - \sigma_j)+(\sigma_j - \sigma_i) - 2 \omega) }\, \, \Gamma_h ( \Delta_X \omega + \sigma_i - \sigma j ) \Gamma_h ( \Delta_X \omega + \sigma_j - \sigma i) \\
 & = e^{- \frac{i \pi} {6 r \omega_1 \omega_2 }  ( 2 \omega (\Delta_x - 1) }\, \, \Gamma_h ( \Delta_X \omega + \sigma_i - \sigma j ) \Gamma_h ( \Delta_X \omega + \sigma_j - \sigma i) 
%%%%%%%%%%%%%%%%%%%%%%%%%%%%%%%%%%%%%
\end{align*}
\begin{align*}
 \lim_{r \rightarrow 0^+ } \Gamma_e\big( {z_i \over z_j})\Gamma_e\big( {z_j \over z_i}) = &e^{- \frac{i \pi} {6 r \omega_1 \omega_2 }  ( (\sigma_j - \sigma_i) + (\sigma_i- \sigma_j) - 2 \omega)} \, \, \Gamma_h (  \sigma_i - \sigma_j) \Gamma_h (  \sigma_j - \sigma_i) =\\
& = e^{- \frac{i \pi} {6 r \omega_1 \omega_2 }  ( - 2 \omega)}  \, \, \Gamma_h (  \sigma_i - \sigma_j) \Gamma_h (  \sigma_j - \sigma_i)
\end{align*}
\begin{align*}
& \lim_{r \rightarrow 0^+} \Gamma_e( 
(pq)^{\frac{\Delta}{2} b y_i z_j }) \Gamma_e( 
(pq)^{\frac{\Delta}{2} b^{-1} \tilde y_i^{-1} z_j^{-1} })  = \\
=& \lim_{r \rightarrow 0^+ } \Gamma_e( e^{2 \pi i  r  {\Delta \over 2} (\omega_1 + \omega_2)} e^{2 \pi i r ( m_i + m_B + \sigma_j) })\Gamma_e( e^{2 \pi i r  {\Delta \over 2} (\omega_1 + \omega_2)} e^{2 \pi i r ( -\tilde m_i - m_B - \sigma_j) }) = \\
= &  \,e^{- \frac{i \pi} {6 r \omega_1 \omega_2 }  ( ( \omega \Delta + m_i + m_B + \sigma_j -  \omega) + ( \omega \Delta -\tilde m_i - m_B - \sigma_j -  \omega))} \, \\
& \Gamma_h ( \omega \Delta + m_i + m_B + \sigma_j) \Gamma_h ( \omega \Delta - \tilde m_i - m_B - \sigma_j) = \\
=&\, e^{- \frac{i \pi} {6 r \omega_1 \omega_2 }  ( 2 \omega (\Delta - 1) + m_i - \tilde m_i )} \, \, \Gamma_h ( \mu_i + \sigma_j) \Gamma_h ( \nu_i - \sigma_j) 
\end{align*}
Dove abbiamo definito le masse reali come
$$
 \mu_i = \omega \Delta + m_i + m_B  \quad  \nu_i = \omega \Delta - \tilde m_i - m_B  
$$
\subsubsection{Contributo divergente}
Il contributo divergente degli esponenziali è uguale a (non scrivo $ \frac{- i \pi}{6 r \omega_1 \omega_2} $):
\begin{align*}
& \, \omega (\Delta_X - 1) ( N_c - 1) + \frac{N_c(N_c -1)}{2} 2 \omega(\Delta_x - 1) - \frac{N_c(N_c -1)}{2}  ( - 2 \omega)  +\\
&+ (N_c - 1  \, \,(\mbox{qualcosa} )) + N_c ( \sum_i^{N_f} 2 \omega(\Delta - 1 ) + m_i - \tilde{m_i} = \\
= &  \omega (\Delta_X - 1) ( N_c^2 - 1) + (N_c^2 -1) \omega + N_c N_f 2 \omega(\Delta - 1) + N_c (\sum_{i}^{N_f}  m_i - \tilde{m_i} )
\end{align*}
Inoltre c'è da considerare anche la misura e la \emph{delta function} nelle nuove coordinate $ z_i = e^{2 \pi i r \sigma_i }$:

$$
\prod_{i=1}^{N_c} \, \frac{dz_i}{2 \pi i z_i} \, \, \delta \big( \prod_{i=1}^{N_c} z_i - 1 \big)
$$ 
Il determinante della trasformazione è
$$
  \det \big (\frac{ \partial z_i }{ \partial \sigma_j})= (2 \pi i r)^{N_c} \prod_{i=1}^{N_c}  z_i \quad \longrightarrow \quad \prod_{i=1}^{N_c} \, \frac{dz_i}{2 \pi i} \, = \, r^{N_c} \prod_{i=1}^{N_c} d \sigma_i
$$
La \emph{delta function} diventa:
\begin{equation}
\delta \big( \prod_{i=1}^{N_c} z_i - 1 \big) \, = \, \delta \big(e^{2 \pi i r \sum \sigma_i} - 1 \big) \quad \longrightarrow \quad  \bigg( \frac{1}{ ((2 \pi i r) e^{2 \pi i r (\sum \sigma_i)}} \bigg)^{N_c} \, \delta ( \sum \sigma_i)
 \end{equation}
\begin{comment}
		 Si può capire il risultato della delta function come un ulteriore cambio di variabile fra le $\sigma_i$:
 	\begin{align*}
 	 \sigma_i \rightarrow \tilde{\sigma_i} \qquad \mbox{per} \qquad i= 1 \ldots N-1 \qquad \sigma_N \rightarrow \tilde{\sigma_N} = S = \sum_{i=1}^{N} \sigma_i \\
 	  Det(J) = Det \bigg( \frac{\partial \sigma_i}{\partial \tilde{\sigma_j} } \bigg) \; = \; 
 	   \begin{pmatrix}[c c c  c|c]
			1 	& 0 		& 0 & \cdots & 0 \\
			0 			 & 1 & 0 & \cdots &  \vdots \\
			\vdots 		& 0 		& \ddots & 0 & \vdots \\
			\vdots & \vdots & 0 & 1 & 0 \\
			\hline 
			1 & \cdots & \cdots & 1  &1 \\
	\end{pmatrix} 
	\end{align*}
\end{comment}

\subsubsection{Funzione di partizione}
Considerando le parti finite, la funzione di partizione diventa
(Manca il limite dei pochhammer)
\begin{align*}
Z_{el} ( \mu_i , \nu_i ) = &
 \frac{1}{ (2 \ \pi i)^{N_c} }{1 \over N_c ! } 
\Gamma_h ( \Delta_X \omega ; \omega_1 , \omega_2)^{N_c-1}
\int_{T^{N_c-1}}  
\prod_{i=1}^{N_c} d \sigma_i \, \delta( \sum_i \sigma_i) \, \\
 & e^{- 2 \pi i r N_c \sum \sigma_i} 
 \prod_{ 1 \leq i<j \leq N_c} \frac{ \Gamma_h( \Delta_X \omega \pm (\sigma_i - \sigma_j)) }{ \Gamma_h ( \pm (\sigma_i - \sigma_j) )} 
 \prod_{i=1}^{N_f} \prod_{j=1}^{N_c} \Gamma_h ( \mu_i + \sigma_j) \Gamma_h ( \nu_i - \sigma_j)
\end{align*}
usando la convenzione $ \Gamma_h ( \pm x ) = \Gamma_h (x) \Gamma_h( -x)$

Avendo compattificato una direzione, il superpotenziale $\eta$ impone la seguente condizione sulle masse reali (in 4D, viene portata anche in 3D):
\begin{equation}
  { 1 \over 2 } \sum_a \mu_a + \nu_a  = \omega ( N_c \Delta_X + (N_f+1))
  \label{eqn:R-symm Anomaly}
\end{equation}
\subsubsection{Masse reali e flow senza $\eta$}

Assegnano le masse reali come segue si ottiene una rottura di $ SU(N_f+1)^2 \times U(1)_B \, \rightarrow \, SU(N_f)^2 \times U(1)_A \times U(1)_B$:
\begin{align*}
\mu = &
\begin{pmatrix}[c c c c|c]
		m_a + m_A 	& 0 		& \cdots 	& \cdots 	& 0 \\
		0 			 & m_a + m_A & 0 &\cdots & \vdots \\
		\vdots 		& 0 		& \ddots & 0 & \vdots \\
		\vdots & \vdots & 0 & m_a + m_A & 0 \\
		\hline 
		0 & \cdots & \cdots & 0 & m  - m_A N_f\\
\end{pmatrix} \\
+ &
\begin{pmatrix}[c c c c|c]
		m_B + \omega \Delta 	& 0 		& \cdots 	& \cdots 	& 0 \\
		0 			 & m_B + \omega \Delta & 0 &\cdots & \vdots \\
		\vdots 		& 0 		& \ddots & 0 & \vdots \\
		\vdots & \vdots & 0 & m_B + \omega \Delta & 0 \\
		\hline 
		0 & \cdots & \cdots & 0 &  m_B + \omega \Delta_m\\
\end{pmatrix}
\end{align*}
dove valgono le condizioni:
$$
 m_A N_f + \sum m_a = 0  %% \quad m = m_A N_f 	
$$
\begin{comment}
Notare come si giunge così a:
$$
m  + \sum_a^{N_f} m_a   = tr_{SU(N_f+1)} m_i = 0 \quad \mbox{dove} \quad m = m_{N_f+1}
$$
Ottenendo così i vincoli che la rottura di $SU(N_f+1) \rightarrow SU(N_f) \times U(1)$.\\
\end{comment}
La stessa cosa si fa per $SU(N_f)_R$ \\

\begin{align*}
\nu = &
\begin{pmatrix}[c c c c|c]
		\tilde m_a + m_A 	& 0 		& \cdots 	& \cdots 	& 0 \\
		0 			 & \tilde m_a + m_A 	& 0 &\cdots & \vdots \\
		\vdots 		& 0 		& \ddots & 0 & \vdots \\
		\vdots & \vdots & 0 & \tilde m_a + m_A 	 & 0 \\
		\hline 
		0 & \cdots & \cdots & 0 & -m  - m_A N_f\\
\end{pmatrix} \\
+ & 
\begin{pmatrix}[c c c c|c]
		- m_B + \omega \Delta 	& 0 		& \cdots 	& \cdots 	& 0 \\
		0 			 & - m_B + \omega \Delta  & 0 &\cdots & \vdots \\
		\vdots 		& 0 		& \ddots & 0 & \vdots \\
		\vdots & \vdots & 0 & - m_B + \omega \Delta & 0 \\
		\hline 
		0 & \cdots & \cdots & 0 & -m_B + \omega \Delta_m\\
\end{pmatrix}
\end{align*}
le stesse condizioni valgono in questo caso (tildate).\\
Il valore di $m_A$ è uguale sia per particelle \emph{left} e \emph{right}, e genera per questo motivo un $U(1)$ "diagonale".\\
NB: $U(1)_A$ mixa con la R-Symmetry e quindi le R-Cariche vanno modificate e sono diverse fra i primi $N_f$ sapori e l'ultimo ( $\Delta $ e $\Delta_m$).\\

NB: da \citep{Aharony:2013dha} (5.28):
Date masse $m_a$ e $\tilde m_a$ per $Q$ e $\tilde{Q}$, abbiamo:
$$
	m_V = {1 \over 2 }( m_a -  m_a) \quad m_A = {1 \over 2 }( m_a + m_a)
$$

\subsubsection{Limite per $m \rightarrow \infty$}
Per fare il limite $m \rightarrow \infty$ utilizziamo la seguente identità \citep{Aharony:2013dha} (formula 5.25 pag 53 vedi def. 5.14)
\begin{align*}
 &\lim_{ m \rightarrow \infty } \Gamma_h ( \omega \Delta + \sigma_i + M + m) = \\
& \exp \bigg( \mbox{sign} (m) \frac{\pi i }{2 \omega_1 \omega_2} \bigg( [ \omega (\Delta
 -1) + \sigma_i + (m+M)]^2 - \frac{\omega_1^2 + \omega_2^2}{12} \bigg) \bigg)
\end{align*} 
Applicandola ai due termini che hanno il termine di massa che andrà all'infinito otteniamo:
\begin{align*}
\Gamma_h ( \sigma_i + \mu_{N_f+1}(m)) \, = &\, \exp \bigg( \, \mbox{sign}(m) \frac{\pi i}{2 \omega_1 \omega_2 } \big[ [ \omega (\Delta_M - 1) + \sigma_i + \\
+ & ( m + m_B - N_f m_A)]^2 - \frac{ \omega_1^2 + \omega_2^2 }{12} \big] \bigg )\\
\Gamma_h ( - \sigma_i + \nu_{N_f+1}(m)) \, = &\, \exp \bigg( \, \mbox{sign}(-m) \frac{\pi i}{2 \omega_1 \omega_2 } \big[ [ \omega (\Delta_M - 1) - \sigma_i + \\
&( - m - m_B - N_f m_A)]^2 - \frac{ \omega_1^2 + \omega_2^2 }{12} \big] \bigg )\\
\end{align*}
Per via del fattore $\sign(\pm m)$ i quadrati dei vari termini si cancellano (termini che darebbero vita a termini \emph{Chern-Simons})
[Se si integrasse un numero diverso di fermioni L o R, avremmo per l'appunto questi termini]. Rimangono solo \emph{alcuni} doppi prodotti. I termini rimanenti sono:
\begin{align*}
\exp \bigg[ \frac{\pi i}{2 \omega_1 \omega_2 } \bigg[& 4 \,  \omega (\Delta_M - 1) ( m + \sigma_i + m_B )  - 4 ( m_a N_f) ( \sigma_i + m + m_B) \bigg] \bigg]  = \\
 = \exp \bigg[ \frac{\pi i}{2 \omega_1 \omega_2 } \bigg[& 4  ( m  + m_B ) ( \omega (\Delta_M - 1) -  m_a N_f ) + 	4 \sigma_i  (\omega (\Delta_M - 1) -  m_a N_f)  \bigg] \bigg]
\end{align*}
Inserendoli all'interno della funzione di partizione si ottiene:
\begin{align*}
 &\prod_{i=1}^{N_c} \exp \bigg[ \frac{\pi i}{2 \omega_1 \omega_2 } \bigg[ 4  ( m  + m_B ) ( \omega (\Delta_M - 1) -  m_a N_f ) + 	4 \sigma_i  (\omega (\Delta_M - 1) -  m_a N_f)\bigg] \bigg] = \\
 &\exp \bigg[ \frac{\pi i}{2 \omega_1 \omega_2 } \bigg[ 4 N_c  ( m  + m_B ) ( \omega (\Delta_M - 1) -  m_a N_f ) +4\big( \sum_{i=1}^{N_c}	 \sigma_i  \big) \big(\omega (\Delta_M - 1) -  m_a N_f \big)  \bigg] \bigg] \\
\end{align*}
Il primo e l'ultimo termine possono esser portati fuori dall'integrale, mentre il termine proporzionale a $ \sum_{i=1}^{N_c}	\sigma_i  $ può essere inglobato nella $\delta( \sum \sigma_i)$.\\
Definendo la funzione $ c(x) = e^{ \frac{i \pi x }{2 \omega_1 \omega_2}}$ i contributi diventano:
$$
c( 4 N_c  ( m  + m_B ) ( \omega (\Delta_M - 1) -  m_a N_f )\,  c( 4\big( \sum_{i=1}^{N_c}	 \sigma_i  \big) \big (\omega (\Delta_M - 1) -  m_a N_f \big)
$$
Utilizziamo la condizione \ref{eqn:R-symm Anomaly} utilizzando questa assegnazione delle masse, ossia:
\begin{equation}
  { 1 \over 2 } \sum_a \mu_a + \nu_a  = \omega ( N_c \Delta_X + N_f + 1 )  \, = \, N_f \omega \Delta + \omega \Delta_M 
\end{equation}
Utilizziamo questa relazione nell'esponenziale precedente (in modo da non avere termini che dipendono dal campo che ha massa che tende a infinito):
\begin{align*}
c( 4 N_c  ( m  + m_B ) ( \omega (1 - \Delta)  - N_c \Delta_X -  m_a N_f )\,  c( 4\big( \sum_{i=1}^{N_c}	 \sigma_i  \big) \big ( \omega( (1 - \Delta)  - N_c \Delta_X  -  m_a N_f ) \big)\\
c( 4 N_c  ( m  + m_B ) (-\omega ( \Delta-1)  +  N_c \Delta_X  +  m_a N_f )\,  c( 4\big( \sum_{i=1}^{N_c}	 \sigma_i  \big) \big ( - \omega (( \Delta -1 )  + N_c \Delta_X  +  m_a N_f  ) \big)\\
\end{align*}

\section{ Teoria Magnetica}

L'indice superconforme a singolo stato per la teoria magnetica è dato dalla seguente espressione ( $\tilde N_c = k N_f - N_c$):
È DIVERSO DA DOLAN-OSBORN--> FLAVOR DEI QUARK SBAGLIATO: $ y -> y^-1 $ per il primo quark\\
 E MESONI SBAGLIATI (riscritto l'indice dalla definizione)
\begin{align*}
 i_M &(p,q, \tilde v, y,\tilde y ,\tilde z) =  \\ 
& -\bigg({p \over 1-p}+{q \over 1-q} -{1 \over (1-p)(1-q)} \big((p\,q)^{s}- (p\,q)^{1-s}\big)
\bigg) \big( p_{\tilde N_c}( \tilde z)\, p_{\tilde N_c}(\tilde z^{-1})-1\big ) + \\ 
& +{1\over (1-p)(1-q)}\bigg((p\,q)^{{1 \over 2 } r} \,\tilde v \, p_{N_f}(y^{-1})\, p_{\tilde N_c}(\tilde z)
- (p\,q)^{1-  {1\over 2}r} \, {1 \over \tilde v}\, p_{{N_f}}(y)\, p_{{\tilde N_c}}(\tilde z^{-1}) + \\
& + (p\,q)^{{1 \over 2 } r}\, {1\over \tilde v}\, p_{{N_f}}({\tilde y}\,) \, p_{{\tilde N_c}}(\tilde z^{-1})
- (p\,q)^{1-{1\over 2}r} \, \tilde v \, p_{N_f}({\tilde y}^{-1})\, p_{\tilde N_c}(\tilde z)\bigg) +  \,\\
&\sum_{l=0}^{k-1}   \bigg( (pq)^{ {1 \over 2 } 2 ( r + l s  )} p_{N_f}(y) p_{N_f}(\tilde y^{-1}) - (pq)^{1 -  {1 \over 2 } 2 ( r + s l )} p_{N_f}(y^{-1} p_{N_f}{\tilde y}\bigg)  \\
\end{align*}

Esplicitando i polinomi otteniamo:

\begin{align*}
 i_M &(p,q, \tilde v, y,\tilde y ,\tilde z) = \\ 
& -\bigg({p \over 1-p}+{q \over 1-q} -{1 \over (1-p)(1-q)} \big((p\,q)^{s}- (p\,q)^{1-s}\big)
\bigg) \big( \sum_{i,j}^{\tilde N_c} \tilde z_i \tilde z_j^{-1} -1\big ) + \\ 
%% CHIRALS
& +{1\over (1-p)(1-q)}\bigg[ \sum_i^{N_f}\sum_j^{\tilde N_c} \bigg( (p\,q)^{{1 \over 2 } r} \,\tilde v \, y_i^{-1}\, \tilde z_j
- (p\,q)^{1- {1\over 2}r} \, {1 \over \tilde v}\, y_i \,   
\tilde z_j^{-1} + \\
& + (p\,q)^{{1 \over 2 } r}\, {1\over \tilde v}\, 
({\tilde y_i}\,) \,  
(\tilde z_j^{-1})
- (p\,q)^{1-  {1\over 2}r} \, \tilde v \, 
{\tilde y_i}^{-1} \, 
\tilde z_j \bigg) + \\
%% MESONS
 &\sum_i^{N_f} \sum_j^{N_f} \sum_{l=0}^{k-1}   \bigg(  (pq)^{ r + s l  } y_i \tilde y_j^{-1}   - (pq)^{1 -( r + s l )}
y_i^{-1}  {\tilde y_i}  \bigg) \bigg] \\
\end{align*}
La prima riga è identica alla teoria elettrica (eccetto per il numero di colori).
Il contributo dai cambi chirali è diverso, essendo le cariche nella nuova teoria diverse.
L'ultima riga come è diversa come struttura dalla teoria elettrica, infatti è il contributo all'indice dai mesoni ( solo flavour nessun colore).
\subsubsection{Contributo dei campi chirali}
Definiamo i seguenti cambi di variabile per calcolare più facilmente l'indice.
$$
	\alpha = {1 \over 2 } r  
$$
	L'indice superconforme può essere così riscritto come:
\begin{align*}
 i_M^{Chiral}  (p,q, \tilde v, y,\tilde y ,\tilde z) = &
 +{1\over (1-p)(1-q)} \sum_i^{N_f}\sum_j^{\tilde N_c} \bigg( (p\,q)^{\alpha} \,\tilde v \, y_i\, \tilde z_j
- (p\,q)^{1- \alpha} \, {1 \over \tilde v}\, y_i^{-1} \,   
\tilde z_j^{-1} + \\
& + (p\,q)^{\alpha}\, {1\over \tilde v}\, 
({\tilde y_i}\,) \,  
(\tilde z_j^{-1})
- (p\,q)^{1- \alpha} \, \tilde v \, 
{\tilde y_i}^{-1} \, 
\tilde z_j \bigg) + \\
\end{align*}
Esponenziamo separatemente i primi due termini dagli ultimi due:
\begin{align*} 
&\exp \bigg( \sum_{n}^{\infty} \frac{1}{n} i_M^{Chiral} (p^n,q^n, \tilde v^n, y^n,\tilde y^n ,\tilde{z^n} ) \bigg)  = \\
& \exp \bigg( \sum_{n}^{\infty} \sum_i^{N_f}\sum_j^{\tilde N_c}  \frac{1}{n}    {1\over (1-p^n)(1-q^n)} \big( (p \, q)^{\alpha n} \,\tilde v^n \, y_i^n\, \tilde z_j^n
- (p\,q)^{(1- \alpha)n} \, \tilde v^{-n}\, y_i^{-n} \,   
\tilde z_j^{-n} \big) \bigg) \\
& \; \mbox{ facciamo il cambio di variabile } \, (pq)^{\alpha} \tilde{v} y_i^{-1} \tilde{z_j} = t \\
=& \prod_i^{N_f}\prod_j^{\tilde N_c}  \exp \bigg( \sum_{n}^{\infty} \frac{1}{n}   {1\over (1-p^n)(1-q^n)} \big( t^n - \big( \frac{pq}{t} \big)^n \big) \bigg) \bigg)
\end{align*}
A questo punto si utilizza l'identità \ref{Gamma_eChiral} e otteniamo così:
\begin{equation}
	\prod_i^{N_f}\prod_j^{\tilde N_c} \Gamma_e ( t ; p,q) = \prod_i^{N_f}\prod_j^{\tilde N_c} \Gamma_e ( (pq)^{\alpha} \tilde{v} y_i^{-1} \tilde{z_j} ; p,q)
\end{equation}
Gli altri due termini si ottengono nello stesso modo, ma facendo il cambio di variabile:
$$
	\frac{1}{\tilde v } \tilde{y_i} z_j^{-1} (pq)^{\alpha} = t'
$$
Ottenendo così il contributo completo dei campi chirali (ripristinando le R-cariche al posto di $\alpha$):
\begin{equation}
\prod_i^{N_f}\prod_j^{\tilde N_c} \Gamma_e ( (pq)^{{1 \over 2 } r} \tilde{v} y_i^{-1} \tilde{z_j} ; p,q)\Gamma_e ( (pq)^{{1 \over 2 } r} \tilde{v}^{-1} \tilde y_i \tilde{z_j}^{-1} ; p,q)
\end{equation}

\subsubsection{Contributo dei mesoni}
L'indice di singolo stato dei mesoni è calcolto come i campi chirali, tenendo conto che la loro R-carica è $ R_M = 2 R_Q + R_X j$ dove $j$ indica l'esponente dell'aggiunta nel mesone.
Utilizzo $m_{ij} = y_i \tilde{y_j}$
(Seconda formula è da Dolan Osborn, inutilmente complicata, equivale alla mia (la prima))
\begin{align*}
 i_M^{Mesons} (p,q, \tilde v, y,\tilde y ,\tilde z) = 
\,  &\frac{1}{(1-p)(1-q)} \sum_i^{N_f} \sum_j^{N_f}   \sum_{l=0}^{k-1}  \bigg( (pq)^{( r+ l s  )}m_{ij}  - (pq)^{1 -( r + l s )}
m_{ij}^{-1} \bigg) = \\ 
= \, & \frac{1}{(1-p)(1-q)}\frac{1 - (pq)^{1-s}}{1 - (pq)^s}  \sum_i^{N_f} \sum_j^{N_f}  \bigg( (pq)^r
m_{ij} - (pq)^{2s-r} 
m_{ij}^{-1} \bigg)\\
\end{align*}

L'esponenziale da calcolare è:
\begin{align*}
& \exp \bigg( \sum_{n}^{\infty} \frac{1}{n} i_M^{Meson} (p^n,q^n, \tilde v^n, y^n,\tilde y^n) \bigg) = \\
& \prod_i^{N_f} \prod_j^{N_f}  \prod_{l=0}^{k-1} \exp \sum_{n}^{\infty} \bigg( \frac{1}{n}  \frac{1}{(1-p^n)(1-q^n)} \bigg( (pq)^{(r + l s )n}
m_{ij}^n - (pq)^{(1-(r + l s )n} 
m_{ij}^{-n} \bigg) \bigg) \\
\end{align*}

Ponendo ora $ (pq)^{r+sl} m_{ij} = y$:
\begin{align*}
& \exp \bigg( \sum_{n}^{\infty} \frac{1}{n} i_M^{Meson} (p^n,q^n, \tilde v^n, y^n,\tilde y^n) \bigg) = \\
& \prod_i^{N_f} \prod_j^{N_f}  \prod_{l=0}^{k-1} \exp \sum_{n}^{\infty} \bigg( \frac{1}{n}  \frac{1}{(1-p^n)(1-q^n)} \bigg( y^n
m_{ij}^n - (pq/y)^{n} 
m_{ij}^{-n} \bigg) \bigg)\\
= \, & \prod_i^{N_f} \prod_j^{N_f}  \prod_{l=0}^{k-1} \Gamma_e ( (pq)^{r + l s} ; p ,q)
\end{align*}

\subsection{Indice e funzione di partizione}

\subsubsection{Espressione dell'indice}
L'indice superconforme per la teoria magnetica è dato da ( $\tilde{N_c} = k N_f - N_c$):
\begin{align*}
&I_{Mag} ( p,q,y,\tilde y,\tilde v) = \\
 & { 1 \over \tilde{N_c !}} (p;p)^{\tilde{N_c}-1} ( q;q)^{\tilde{N_c}-1} \, \Gamma_e( (pq)^s;p,q)^{\tilde{N_c}-1} \bigg( \prod_i^{N_f} \prod_j^{N_f}  \prod_{l=0}^{k-1} \Gamma_e ( (pq)^{r + l s} y_i \tilde{y_j}^{-1}; p ,q) \bigg) \\
&\int_{T^{\tilde{N_c-1}}} \bigg( \prod_{i=1}^{ \tilde{N_c }} { d z_i \over { 2 \pi i z_i} } \bigg) \delta \bigg( \prod_{i=1}^{\tilde{N_c}} z_i - 1 \bigg) 
\prod_{ 1 \leq i < j \leq \tilde{N_c}} \frac{ \Gamma_e \big( (pq)^s {z_i \over z_j} \big) \Gamma_e \big( (pq)^s {z_j \over z_i} \big) }{ \Gamma_e( {z_i \over z_j};p,q) \Gamma_e({z_j \over z_i}; p,q)} \\
& \prod_{ 1 \leq j \leq \tilde{N_c}} \prod_{ 1 \leq i \leq N_f} \Gamma_e ( (pq)^{{1 \over 2 } r} \tilde{v} y_i^{-1} \tilde{z_j} ; p,q)\Gamma_e ( (pq)^{{1 \over 2 } r} \tilde{v}^{-1} \tilde y_i \tilde{z_j}^{-1} ; p,q)
\end{align*}
dove $r $ è la R-Carica del quark duale e $ s = \frac{1}{k+1} = {1 \over 2 } \Delta$.

\subsubsection{Funzione di partizione}
Come fatto per la teoria elettrica si riduce l'indice superconforme alla funzione di partizione della teoria.
I contributi del campo vettoriale e del campo chirale nell'aggiunta sono espressioni identiche ( in funzione del nuovo numero di colori ($\tilde{N_c}$).\\
\subsubsection*{Mesoni}
Il contributo dei mesoni è dato da:

\begin{align*}
&\prod_i^{N_f} \prod_j^{N_f}  \prod_{l=0}^{k-1} \Gamma_e ( (pq)^{r + l s} y_i \tilde{y_j}^{-1}; p ,q)  = \\
& \prod_i^{N_f} \prod_j^{N_f}  \prod_{l=0}^{k-1} \Gamma_e \big ( \exp( 2 \pi i r [(2 \omega)(r + l s) +  ( m_i - \tilde{m_j})]); p ,q \big)  = \\
\overset{ r\rightarrow 0 }{ \sim } &  \prod_i^{N_f} \prod_j^{N_f}  \prod_{l=0}^{k-1} \exp \bigg( \frac{- i \pi }{6 r \omega_1 \omega_2} \big( 2 \omega ( r + l s ) + m_i - \tilde{m_j} - \omega \big) \bigg) \Gamma_h \big( \omega ( 2 \Delta + \Delta_X) + m_i + \tilde{m_j} \big)   = \\
= &  \exp \bigg( \frac{- i \pi }{6 r \omega_1 \omega_2} \big(  N_f^2  \bigg( \sum _{l=0}^{k-1} 2 \omega ( r + l s -1  )\bigg)  +N_f \big( \sum_i^{N_f} m_i - \tilde{m_i}\big) \big) \bigg) \\
&\qquad \qquad  \prod_i^{N_f} \prod_j^{N_f}  \prod_{l=0}^{k-1}  \Gamma_h \big( \omega ( 2 \Delta + \Delta_X) + m_i + \tilde{m_j} \big)  
\end{align*}
%% dove il vincolo sulle masse reali $ \sum m_i = \sum \tilde{m_j} = 0$ le ha eliminate dal fattore divergente.\\

\subsubsection*{Chirali fondamentali}
Per i chirali nella fondamentale abbiamo ( $r = \Delta$):

\begin{align*}
 \prod_{ 1 \leq j \leq \tilde{N_c}} \prod_{ 1 \leq i \leq N_f} & \Gamma_e ( (pq)^{{1 \over 2 } r} \tilde{v} y_i^{-1} \tilde{z_j} ; p,q)\Gamma_e ( (pq)^{{1 \over 2 } r} \tilde{v}^{-1} \tilde y_i \tilde{z_j}^{-1} ; p,q)  = \\
 = \prod_{ 1 \leq j \leq \tilde{N_c}} \prod_{ 1 \leq i \leq N_f}  & \Gamma_e ( \exp\big( 2 \omega( {1 \over 2 } r )   + \tilde{m_B} - m_i  + \tilde{\sigma_j} \big) ; p,q) \\
 &\Gamma_e ( \exp\big( 2 \omega(  {1 \over 2 } r )  - \tilde{m_B} + \tilde  m_i  - \tilde{\sigma_j} \big) ; p,q)  = \\
 \overset{ r \rightarrow 0 } { \sim } \prod_{ 1 \leq j \leq \tilde{N_c}} \prod_{ 1 \leq i \leq N_f}   &\exp \bigg( \frac{- i \pi }{6 r \omega_1 \omega_2} \bigg(  \big(\omega ( \Delta - 1) + \tilde{m_B} - m_i  + \tilde{\sigma_j} \big)  +  \big( \omega (  \Delta - 1) - \tilde{m_B} + \tilde m_i  - \tilde{\sigma_j}  \big)\bigg) \\
&\Gamma_h \big(\omega \Delta + \tilde{m_B} + m_i  + \tilde{\sigma_j} \big) \Gamma_h \big(\omega \Delta - \tilde{m_B} +  \tilde m_i  - \tilde{\sigma_j} \big)  = \\
 =   \exp \bigg( \frac{- i \pi }{6 r \omega_1 \omega_2} &\big( 2  N_f \tilde{N_c}\omega (\Delta - 1 ) +  \tilde{N_c} \big( \sum_{i=1}^{N_f} - m_i + \tilde{m_i} \big)  \big) \bigg)  \Gamma_h \big( \mu_i + \tilde{\sigma_j} \big) \Gamma_h \big(\nu_i - \tilde{\sigma_j} \big)  
\end{align*}
Dove abbiamo definito le masse reali 
$$
 \mu_i = \omega \Delta + \tilde{m_B} + m_i   \qquad \nu_i  = \omega \Delta - \tilde{m_B} +  \tilde m_i 
$$


\subsubsection*{Campo di gauge e materia nell'aggiunta}
Come detto precedentemente i contributi sono come nel caso elettrico:
\begin{align*}
& \Gamma_e((pq)^{\frac{\Delta_X}{2}})^{N_c -1} =
\Gamma_e( e^{2 \pi i r  {\Delta_X \over 2} (\omega_1 + \omega_2)})^{N_c-1} =  \Gamma_e( e^{2 \pi i r  \Delta_X \omega })^{N_c-1} = \\
&=\big[ e^{- \frac{i \pi} {6 r \omega_1 \omega_2 }  ( \omega \Delta_x - \omega) }\big]^{N_c-1} \Gamma_h ( \omega \Delta_X ; \omega_1, \omega_2)^{N_c-1}
%%%%%%%%%%%%%%%%%%%%%%%%
\end{align*}
 \begin{align*}
 & \Gamma_e\big( (pq)^{\frac{\Delta_X}{2}} \big( { z_i \over z_j} \big) \big) \Gamma_e \big( (pq)^{\frac{\Delta_X}{2}} \big( { z_j \over z_i} \big) \big) = \\
&= \Gamma_e\big( e^{2 \pi i r \frac{\Delta_X}{2} (\omega_1 + \omega_2)} e^{ 2 \pi  i r (\sigma_i - \sigma j)} \big) \big) \Gamma_e \big( e^{ 2 \pi i r \frac{\Delta_X}{2} (\omega_1 + \omega_2)}  e^{ 2 \pi i r (\sigma_j - \sigma_i)} \big) =  \\
&= \Gamma_e\big( e^{2 \pi  i r (\Delta_X \omega +\sigma_i - \sigma j} \big) \Gamma_e \big(  e^{2 \pi  i r (\Delta_X \omega +\sigma_j - \sigma_i} \big) \\
 & = e^{- \frac{i \pi} {6 r \omega_1 \omega_2 }  ( 2 \omega \Delta_x + (\sigma_i - \sigma_j)+(\sigma_j - \sigma_i) - 2 \omega) }\, \, \Gamma_h ( \Delta_X \omega + \sigma_i - \sigma j ) \Gamma_h ( \Delta_X \omega + \sigma_j - \sigma i) \\
 & = e^{- \frac{i \pi} {6 r \omega_1 \omega_2 }  ( 2 \omega (\Delta_x - 1) }\, \, \Gamma_h ( \Delta_X \omega + \sigma_i - \sigma j ) \Gamma_h ( \Delta_X \omega + \sigma_j - \sigma i) 
%%%%%%%%%%%%%%%%%%%%%%%%%%%%%%%%%%%%%
\end{align*}
\begin{align*}
  \Gamma_e\big( {z_i \over z_j})\Gamma_e\big( {z_j \over z_i}) = &e^{- \frac{i \pi} {6 r \omega_1 \omega_2 }  ( (\sigma_j - \sigma_i) + (\sigma_i- \sigma_j) - 2 \omega)} \, \, \Gamma_h (  \sigma_i - \sigma_j) \Gamma_h (  \sigma_j - \sigma_i) =\\
& = e^{- \frac{i \pi} {6 r \omega_1 \omega_2 }  ( - 2 \omega)}  \, \, \Gamma_h (  \sigma_i - \sigma_j) \Gamma_h (  \sigma_j - \sigma_i)
\end{align*}
Attenzione che l'ultimo termine entra nella funzione di partizione al denominatore.\\

\subsection{Contributi divergenti}
Cerchiamo ora di mettere insieme tutti i contributi divergenti ottenuti dal limite per $ r\rightarrow 0$.
Essendo le 't Hooft anomalies per anomalie gravitazionali, devono matchare con il corrispettivo elettrico.\\
Scriviamo solo l'esponente ( a meno di $ \frac{- i \pi }{6 r \omega_1 \omega_2}$):\\
C'E' DA FARE ANCORA IL CONTRIBUTO DEI POCHHAMMER!!
\begin{align*}
& \overbrace{ (\tilde N_c - 1) \hbox{Q-pochhammer} + ( 2 \omega) \frac{\tilde N_c(\tilde  N_c-1)}{2}}^{\hbox{Adj Vector}} +  \overbrace{ ( \omega (\Delta_x - 1)(\tilde  N_c - 1) + ( 2 \omega (\Delta_x - 1)\frac{\tilde  N_c(\tilde N_c-1)}{2}) }^{ \hbox{Adj Chiral}} + \\
& \overbrace{2  N_f \tilde{N_c}\omega (\Delta - 1 ) +  \tilde{N_c} \big( \sum_{i=1}^{N_f} - m_i + \tilde{m_i} \big) }^{\hbox{Fond Chirals}} +  \overbrace{ N_f^2 \big(  \sum _{l=0}^{k-1} 2 \omega ( \Delta + l \frac{\Delta_X}{2} -1 ) \big)  + N_f \big( \sum_i^{N_f} m_i - \tilde{m_i} \big)}^{\hbox{Mesons}} = \\
= & \overbrace{ (\tilde N_c - 1) \hbox{Q-pochhammer} + \omega \tilde  N_c( \tilde N_c-1)}^{\hbox{Adj Vector}} +  \overbrace{ ( \omega (\Delta_x - 1)(\tilde N_c^2 - 1)}^{ \hbox{Adj Chiral}} + \\
& \overbrace{2  N_f \tilde{N_c}\omega (\Delta - 1 ) +  \tilde{N_c} \big( \sum_{i=1}^{N_f} - m_i + \tilde{m_i} \big) }^{\hbox{Fond Chirals}} +  \overbrace{ N_f^2 \big(  \sum _{l=0}^{k-1} 2 \omega ( \Delta + l \frac{\Delta_X}{2}-1 ) \big)  + N_f \big( \sum_i^{N_f} m_i - \tilde{m_i} \big)}^{\hbox{Mesons}}
\end{align*}
Per calcolare il contributo dato dai mesoni è sufficiente notare che:
$$
 \sum_{i=0}^{n} i =  \frac{n ( n + 1 )}{2}
$$
Nel caso del mesone, la somma va fino a $ k-1$:
$$
N_f^2 2 \omega ( \Delta -1) k + \omega \Delta_X  \frac{ ( k-1) k }{2} + N_f \big( \sum_i^{N_f} m_i - \tilde{m_i} \big)
$$
\begin{align*}
= & \overbrace{ (\tilde {N_c}  - 1) \hbox{Q-pochhammer} + \omega \tilde  N_c( \tilde N_c-1)}^{\hbox{Adj Vector}} +  \overbrace{ ( \omega (\Delta_x - 1)( \tilde N_c^2 - 1)}^{ \hbox{Adj Chiral}} + \\
& \overbrace{2  \tilde N_f N_c \omega (\Delta - 1 ) +  \tilde{N_c} \big( \sum_{i=1}^{N_f} - m_i + \tilde{m_i} \big) }^{\hbox{Fond Chirals}} +  \overbrace{N_f^2 2 \omega ( \Delta -1) k + \omega \Delta_X  \frac{ ( k-1) k }{2} + N_f \big( \sum_i^{N_f} m_i - \tilde{m_i} \big)}^{\hbox{Mesons}}
\end{align*}

\bibliography{bibliografia}
%%\bibliographystyle{plainnat}
\end{document}
 
