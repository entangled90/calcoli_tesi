\documentclass[a4paper,12pt]{report}
\usepackage[italian]{babel}
\usepackage{amsmath}
%%\usepackage{mathtools}
\usepackage{amstext}
\usepackage{amssymb}
\usepackage{amsthm}
\usepackage{fullpage}
%%\usepackage{mathrsfs}
\usepackage{natbib}
\usepackage[utf8]{inputenc}


\begin{document}
L'identità scritta in ( cfr. \citep{vanDeBult:2007} pag 30) è :
\begin{align}
&\lim_{r \rightarrow 0^+} \Gamma_e (e^{ 2 i r z}; e^{ i  r \omega_1} , e^{i r  \omega_2})
 e^{\frac{ i \pi^2 }{12  r \omega_1 \omega_2 } ( 2 z - \omega_1 -\omega_2)} =\\
&\lim_{r \rightarrow 0^+} \Gamma_e (e^{ 2 i r z}; e^{ i  r \omega_1} , e^{i r  \omega_2}) 
 e^{\frac{ i \pi^2 }{6 r \omega_1 \omega_2 } (  z - \omega )} = \Gamma_h(z;\omega_1 , \omega_2)
\end{align}
con $ \omega = {1 \over 2 } ( \omega_1 + \omega_2)$.
Si possono riscalare le variabili in modo da sistemare il fattore di $\pi$ all'esponente:
$$
 z \rightarrow \pi z \quad  \omega_1 \rightarrow  \pi \omega_1 \quad  \omega_2 \rightarrow  \pi \omega_2 
$$
Si ottiene:
\begin{align*}
&\lim_{r \rightarrow 0^+} \Gamma_e (e^{ 2 i \pi r z}; e^{ i \pi  r \omega_1} , e^{i \pi r  \omega_2}) =
 e^{\frac{ - i \pi^2 }{6 r  (\pi \omega_1) (\pi \omega_2) } ( (\pi z) - (\pi \omega))}\Gamma_h(\pi z; \pi \omega_1 ,\pi \omega_2) =   e^{\frac{ - i \pi }{6 r \omega_1 \omega_2 } (  z - \omega )} \, \Gamma_h ( z; \omega_1 , \omega_2 ) \\
\end{align*}
Considerando la proprietà di rescaling di $\Gamma_h$: la sua definizione infatti è ( cft \citep{vanDeBult:2007} 2.2.4):
\begin{align}
 \Gamma_h ( z;\omega_1, \omega_2) =& \exp \bigg( \pi i \frac{(2z-\omega_1 - \omega_2)^2}{8 \omega_1 \omega_2 } - \pi i \frac{(\omega_1^2 + \omega_2^2)}{ 24 \omega_1 \omega_2} \bigg) \\
 & \frac{ (\exp( -2 \pi i (z-\omega_2)/ \omega_1 ); \exp( 2 \pi i \omega_2 / \omega_1 ))_{\infty}}
 { (\exp( -2 \pi z / \omega_2 ); \exp( - 2 \pi i \omega_1 / \omega_2))_{\infty}}\\
 & = \Gamma_h ( \pi z; \pi \omega_1, \pi \omega_2)
\end{align}
L'identità utilizzata nel tuo articolo ( \citep{Amariti:2014iza}) e in \citep{Aharony:2013dha} invece ha un fattore 2 di differenza sulle $\omega_i$ nell'argomento della $\Gamma_e$:
$$
\lim_{r \rightarrow 0^+} \Gamma_e (e^{ 2 i \pi r z}; e^{ i 2 \pi  r \omega_1} , e^{i  2 \pi r  \omega_2}) = e^{- \frac{i \pi}{6 \omega_1 \omega_2 r } ( z - \omega)} \Gamma_h(z; \omega_1, \omega_2)
$$
La $\Gamma_h$ non mi sembra essere invariante rispetto a un rescaling delle $\omega_i$, lasciando invariata z. \\
Inoltre la struttura dell'esponente $ \sim (z-\omega)$ deve rimanere tale, altrimenti il coefficiente di fronte alla parte divergente (generato da $(z-\omega)$), proporzionale alle anomalie) non si annulla.\\
Per questo sono portato a pensare che la formula di van de Bult sia sbagliata di un fattore 2.
Puoi confermare?

\bibliography{bibliografia}
\bibliographystyle{plainnat}

\end{document}